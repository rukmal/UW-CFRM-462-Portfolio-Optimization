\usepackage{amsthm}
\usepackage{amsmath}
\usepackage{amssymb}
\usepackage{bm}
\usepackage{caption}
\usepackage{enumerate}
\usepackage{graphicx}
\usepackage{hyperref}
\usepackage{listings}
\usepackage{parskip}
\usepackage{pgfplots}
\usepackage{relsize}
\usepackage{tikz}

\usepgfplotslibrary{fillbetween}
\setcounter{secnumdepth}{0}
\allowdisplaybreaks[1]

\newcommand{\doubleu}[1]{\underline{\underline{#1}}}
\newcommand{\ddx}[1]{\frac{d}{dx} \bigg( #1 \bigg)}
\newcommand{\partiald}[2]{\frac{\partial #1}{\partial #2}}
\newcommand{\matrixpartiald}[2]{\dfrac{\partial #1}{\partial #2}}
\newcommand{\mathematica}[1]{Mathematica: \texttt{#1}}
\newcommand{\intbyparts}[7]{
	\begin{aligned}
		\text{Let}\   #1 &= #4      &      &\text{and}\    &    \frac{d#2}{d#3} &= #7 \\
		\Rightarrow \frac{d#1}{d#3} &= #5 \   &  &   &  \Rightarrow #2 &= #6
	\end{aligned}
	\\ \text{We know} \  \int #1 \, d#2 = #1#2 - \int #2 \, d#1 \\
	= {#4 \cdot #6} - \int {#6 \cdot #5} \, d#3
}
\newcommand{\threematrix}[9]{
	\begin{bmatrix}
		#1 & #4 & #7 \\
		#2 & #5 & #8 \\
		#3 & #6 & #9
	\end{bmatrix}
}
\newcommand{\threevector}[3]{
	\begin{bmatrix}
		#1 \\ #2 \\ #3
	\end{bmatrix}
}
